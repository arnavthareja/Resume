% !TEX program = xelatex

%%%%%%%%%%%%%%%%%%%%%%%%%%%%%%%%%%%%%%%%%
% Medium Length Professional CV
% LaTeX Template
% Version 2.0 (8/5/13)
%
% This template has been downloaded from:
% http://www.LaTeXTemplates.com
%
% Original author:
% Trey Hunner (http://www.treyhunner.com/)
%
% Important note:
% This template requires the resume.cls file to be in the same directory as the
% .tex file. The resume.cls file provides the resume style used for structuring the
% document.
%
%%%%%%%%%%%%%%%%%%%%%%%%%%%%%%%%%%%%%%%%%

%----------------------------------------------------------------------------------------
%	PACKAGES AND DOCUMENT CONFIGURATIONS
%----------------------------------------------------------------------------------------

\documentclass{resume} % Use the custom resume.cls style

\usepackage[left=0.5in,top=0.4in,right=0.5in,bottom=0.4in]{geometry} % Document margins
\usepackage{hyperref}
\usepackage{xcolor}
\usepackage{fontawesome5}
\hypersetup{colorlinks,breaklinks,linkcolor=black,urlcolor=black,anchorcolor=black,citecolor=black}

\name{Arnav Thareja} % Your name
\address{\faIcon{phone-alt}\hspace{0.3em} 858.252.9415\hspace{0.6em}|\hspace{0.6em}\faIcon{envelope}\hspace{0.3em} \href{mailto:athareja@cs.washington.edu}{athareja@cs.washington.edu}}
\address{\faIcon{globe-americas}\hspace{0.3em} \href{https://arnavthareja.github.io}{arnavthareja.github.io}\hspace{0.6em}|\hspace{0.6em}\faIcon{linkedin}\hspace{0.3em} \href{https://www.linkedin.com/in/arnavthareja/}{linkedin.com/in/arnavthareja}\hspace{0.6em}|\hspace{0.6em}\faIcon{github}\hspace{0.3em}\href{https://www.github.com/arnavthareja}{github.com/arnavthareja}} % Your phone number and email

\begin{document}
\vspace{-0.5em} % Remove extra vertical space

%----------------------------------------------------------------------------------------
%	EDUCATION
%----------------------------------------------------------------------------------------

\begin{rSection}{Education}

\begin{education}{University of Washington}{Expected Graduation: June 2024}{Bachelor of Science, Computer Science}{Seattle, WA}
\item \textbf{Cumulative GPA:} 3.97
\item \textbf{Coursework:} Data Structures \& Parallelism, The Hardware/Software Interface, System \& Software Tools, Discrete Math, Probability and Statistics, Linear Algebra, Differential Equations
\item \textbf{Planned Coursework:} Algorithms, Operating Systems, Distributed Systems, Autonomous Robotics, Computer Vision, \\ Databases, Systems Programming
\end{education}

\end{rSection}

%----------------------------------------------------------------------------------------
%	EXPERIENCE
%----------------------------------------------------------------------------------------

\begin{rSection}{Experience}

\begin{rSubsection}{Personal Robotics Lab}{May 2021 – Present}{Undergraduate Researcher}{Seattle, WA}
\item Working on multi-agent autonomous navigation and task allocation with \href{https://mushr.io}{MuSHR cars}
\item Designed and implemented algorithms for non-holonomic multi-agent navigation with optimal task allocation in C++
\item Built ROS (Robot Operating System) wrappers around algorithms to enable easy interfacing with existing systems
\item Sped up robot trajectory comparison framework by 50x by directly analyzing ROS bags through the rosbag Python API
\item Demonstrated and tested system capabilities and translation to real-world environments on physical robots
\end{rSubsection}

\begin{rSubsection}{Husky Robotics}{October 2020 – Present}{Software Engineer, Autonomous Navigation Subteam}{Seattle, WA}
\item Created robot pathfinding and autonomous navigation algorithms for a prototype Mars rover using C++
\item Integrated ROS2 into codebase using nodes and topics for navigation plan visualization
\item Defined and implemented a navigation algorithm to locate targets based on approximate GPS coordinates
\item Designed patterns for driving between two posts given GPS coordinates of the center
\item Leveraged Docker for CI (continuous integration)
\end{rSubsection}

\begin{rSubsection}{Mathnasium}{May 2019 – June 2020}{Instructor}{Renton, WA}
\item Taught K-12 students topics in math up to calculus and helped develop an intuitive understanding of math concepts
\item Contributed to smooth operation of the center and interacted with parents and prospective customers
\end{rSubsection}

\end{rSection}

%----------------------------------------------------------------------------------------
%	PROJECTS
%----------------------------------------------------------------------------------------

\begin{rSection}{Projects}

\begin{project}{\href{https://github.com/arnavthareja/chess}{Chess}}{\href{https://github.com/arnavthareja/chess}{github.com/arnavthareja/chess}}{Personal Project}
\item Built a chess game in Java that can be played in the terminal
\item Implemented a minimax algorithm with alpha-beta pruning for automated gameplay with informed move selection
\item Used a heuristic-based iterative deepening depth first search algorithm and memoization to improve runtime
\end{project}

\begin{project}{\href{https://devpost.com/software/angles-sqdzlt}{Angles}}{\href{https://devpost.com/software/angles-sqdzlt}{devpost.com/software/angles-sqdzlt}}{DubHacks 2020 – Newsprint Track Finalist (Top 3 out of 70+ Projects)}
\item Developed a Chrome Extension that suggests news articles of opposite bias when a news website is visited
\item Leveraged Google Cloud NLP with JavaScript to extract keywords from news articles to use in our opposite bias algorithm
\item Selected as a finalist in the Newsprint track and recognized as one of the top 3 projects out of over 70 projects
\end{project}

\begin{project}{\href{https://yearbook-hhs.web.app/}{Yearbook 2020}}{\href{https://yearbook-hhs.web.app/}{yearbook-hhs.web.app}}{Personal Project}
\item Designed and developed a web application for students and graduates to sign yearbooks virtually during COVID-19
\item Utilized JavaScript, HTML, CSS, and Google Firebase for user authentication, cloud storage, and NoSQL database
\end{project}

\begin{project}{\href{https://github.com/arnavthareja/clcbs\_ros}{CL-CBS ROS Wrapper}}{\href{https://github.com/arnavthareja/clcbs\_ros}{github.com/arnavthareja/clcbs\_ros}}{Personal Robotics Lab}
\item Created a ROS wrapper and defined a ROS API around the Car-Like Conflict-Based Search (CL-CBS) algorithm using C++
\item Extended CL-CBS functionality to allow for parameter reconfiguration and restriction of allowed motion primitives
\end{project}

\end{rSection}

%----------------------------------------------------------------------------------------
%	SKILLS
%----------------------------------------------------------------------------------------

\begin{rSection}{Skills}
\begin{tabular}{ @{} >{\bfseries}l @{\hspace{6ex}} l }
Languages & Java, C++, Python, C, JavaScript, HTML, CSS \\
Tools & ROS (Robot Operating System), Docker, GDB (GNU Debugger), Linux, CMake, Git, GitHub, LaTeX \\
\end{tabular}

\end{rSection}

\end{document}